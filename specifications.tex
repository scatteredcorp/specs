\documentclass{article}
% Comment the following line to NOT allow the usage of umlauts
\usepackage[utf8]{inputenc}
\usepackage[french]{babel}
% Uncomment the following line to allow the usage of graphics (.png, .jpg)
%\usepackage{graphicx}

% Start the document
\begin{document}
\title{Decentralized Marble Game}
\author{Scattered Corporation}


\maketitle

\newpage
\tableofcontents
\newpage

% Create a new 1st level heading
\section{Introduction}
BilleGateCoin est une plateforme décentralisée pour un jeu de billes.
Le jeu de billes est un jeu classique dont le but est de collectionner des billes plus ou moins rares en combattant des adversaires ou en les échangeant.

\section{Déroulement d'une partie}
Une partie met en jeu les billes des joueurs : celui qui remporte la partie gagne la bille de son adversaire.
La façon la plus classique de jouer au jeu de billes est de lancer sa bille sur celle de l'adversaire pour gagner.

Dans l'implémentation de ce jeu, le joueur crée la partie avec un adversaire sélectionné.
Un terrain de jeu est généré avec des obstacles placés aléatoirement et les deux billes des joueurs sont placées de part et d'autre du terrain. Le joueur lance sa bille en direction de celle de l'adversaire. Si ce joueur touche sa bille, il la remporte. Sinon c'est au tour de l'adversaire de lancer sa bille. 


\section{Décentralisation}
Cette implémentation du jeu de billes sera décentralisée. 
La décentralisation du jeu offre de nombreux avantages.

\subsection{Confiance}
Les joueurs n'ont pas à faire confiance à une autorité centrale. 
\textit{Don't trust, verify.}

Le processus de création des billes ainsi que la définition de leur rareté est entièrement transparent et contrôlé par les joueurs mêmes, plutôt qu'une entreprise privée. Le taux de billes créées est fixé par les joueurs et personne ne peut le modifier sans le consensus général de la communauté.

\subsection{Point of Failure}
Dans un système décentralisé, il n'existe pas de point unique de défaillance (\textit{single point of failure}). 

Dans une plateforme centralisée, si le système est défaillant, tout le réseau est impacté. Contrairement à un système décentralisé où il ne suffit d'un seul ordinateur pour assurer la stabilité du réseau.

\subsection{Ouvert au développement}
N'importe quel individu avec des notions en programmation peut contribuer au développement de la plateforme.
Si assez de joueurs (consensus) acceptent ces changements alors ce code est ajouté à la plateforme.

Rien n'empêche un joueur de cloner cette plateforme afin de créer un jeu avec des règles différentes : \textit{Hard Fork}.

\section{Fonctionnement}
Afin de permettre la décentralisation du jeu, nous utiliserons la technologie de la Blockchain.

La \textit{Blockchain} est un régistre distribué (\textit{distributed ledger}) immuable et sécurisé par des milliers d'ordinateurs.

\subsection{Utilisateurs}
Sur une plateforme de jeu décentralisée, il n'y a pas de système de comptes avec une adresse email et un mot de passe.
Chaque joueur doit alors se générer une clé privée ainsi qu'une clé publique correspondante, avec un algorithme de cryptographie tels que RSA ou \textit{Elliptic Curve Digital Signature Algorithm} (ECDSA).

Cette clé privée doit être précieusement cachée car elle permet de signer toutes les transactions du joueur, c'est-à-dire créer une partie, lancer sa bille, échanger des billes, etc.

\subsection{Lancement de la partie}
Deux modèles sont envisageables pour la création d'une partie.
\subsubsection{Deux étapes}
Le joueur choisit la bille qu'il veut jouer, son adversaire (sa clé publique) et la bille de son adversaire (qui lui appartient). Le joueur signe l'invitation avec sa clé privée et la diffuse sur la blockchain. L'adversaire reçoit alors une invitation pour créer cette partie, s'il accepte la partie, il doit également signer le contrat avec sa clé privée sur la blockchain. La partie commence.

Ce modèle assume que les deux joueurs n'ont pas de moyen de communication direct et utilisent la blockchain comme plateforme pour trouver des joueurs.

Ce modèle est couteux car il demande deux contrats sur la blockchain (création de la partie, et acceptation de la partie).


\subsubsection{Double signature}
Le joueur choisit la bille qu'il veut jouer, son adversaire (sa clé publique) et la bille de son adversaire (qui lui appartient). Le joueur signe ce contrat avec sa clé privée et demande à l'adversaire de signer également ce contrat avec sa clé privée. Une fois le contrat signé par les deux joueurs, le contrat est diffusé sur la blockchain. La partie commence.

Ce modèle assume que les deux joueurs ont un moyen de communication direct.

Ce modèle est moins couteux pour le joueur car l'accord entre les deux joueurs est fait hors-blockchain (\textit{off-chain}).
La création d'une plateforme centralisée permettant aux joueurs de trouver un accord est possible. La centralisation de cette plateforme ne pose pas de problèmes sur l'aspect décentralisé du jeu en lui-même.

\subsubsection*{}
Lors de la création de la partie, les joueurs doivent donner une ou plusieurs billes aux \textit{mineurs} afin de traiter la transaction (\textit{transaction fee}). Cette taxe permet aux mineurs d'être rémunérés pour leur travail.

La bille que le joueur choisit est en réalité le hash d'une \textit{claim transaction} ainsi que son emplacement dans celle-ci (voir section Fin de partie).

\subsection{Source d'entropie}
Lorsque la partie est créée, des obstacles sont aléatoirement placés sur le terrain pour ajouter de la difficulté au jeu de billes. Comme dans la vraie vie, il peut y avoir des trous, des arbres, des obstacles. Comment pouvons-nous générer un nombre aléatoire dans une Blockchain ?
Nous ne pouvons-pas générer de nombres aléatoires sur les nodes car le résultat serait différent pour tout le monde, alors nous devons utiliser une autre source d'entropie.

\subsubsection{Solution}

Lorsque le contrat de la création de la partie est signé, celui-ci est diffusé dans la \textit{mempool} des ordinateurs qui assurent la sécurité du réseau : \textit{nodes}. Les mineurs vont alors essayer de former un block avec les contrats et transactions de la mempool.

Pour se faire, les mineurs vont calculer un hash \textit{SHA-256} en concatenant toutes les valeurs du block telles que la date du block, les données de chaque contrat et l'indice du dernier block. Si ce block contient un certain nombre \textit{(mining difficulty)} de `0` en tête de hash, alors ce hash est valide et le block est dit "miné".

Ce hash n'est qu'un nombre hexadécimal aléatoire compris entre 1 et la \textit{mining difficulty}. Ce nombre va permettre la génération des obstacles la partie, en fonctionnant comme un \textit{seed}.

\subsubsection{Limites}
Toutes les parties créées dans ce block auront alors le même seed, c'est-à-dire l'exacte même répartition des obstacles.  

Il suffit de régler le \textit{block time} de la blockchain à un temps court pour permettre une génération de block plus rapide et donc des parties plus diversifiées. 

\subsection{Déroulement de la partie}
Une fois la partie créée, si le seed de la partie est pair c'est au joueur 1 de commencer, sinon c'est au tour du joueur 2.

Pour lancer sa bille, le joueur doit cliquer sur sa bille, et la tirer avec une certaine force. Un début de trajectoire s'affichera pour que le joueur puisse voir approximativement la direction de sa bille.
Une fois la bille lâchée, un vecteur force est créé. Le joueur doit alors signer ce vecteur force avec sa clé privée. Celui-ci est exécuté sur l'écran du joueur, il voit alors sa bille se déplacer. Le vecteur signé est également diffusé sur la Blockchain.

Les nodes et les mineurs vont vérifier la nouvelle position de la bille en fonction du vecteur signé par le joueur. Si la bille touche celle de l'adversaire alors la partie se termine et il remporte toutes les billes. Dans le cas contraire la partie continue et c'est au tour de l'autre joueur.

Pour éviter des problèmes de précision, la nouvelle position de la bille n'est jamais enregistrée dans la Blockchain. Ainsi pour déterminer l'état actuel de la partie (la position des deux billes) il faut recalculer la position de chaque bille grâce aux précédents vecteurs force signés par les joueurs respectifs.

Plus la partie est longue, plus le calcul de la position de la bille est coûteux en ressources, les taxes de transactions seront alors de plus en plus élevées. Les joueurs doivent se mettre d'accord en début de partie du nombre de tours maximum, si ce nombre est depassé pendant la partie, celle-ci se termine et les joueurs reprennent leur bille.

\subsection{Fin de partie}
Lorsqu'un joueur parvient à toucher la bille de l'autre, celui-ci gagne la partie. Pour récupérer ses gains (la bille de l'adversaire ainsi que sa propre bille), le joueur doit diffuser sur la blockchain une \textit{claim transaction}\footnote{Une claim transaction est comparable à un \textit{UTXO} du Bitcoin. Cette transaction prend un ou plusieurs inputs : la bille de l'adversaire et la mise du joueur, et un ou plusieurs outputs, en général la clé publique de celui qui reçoit les gains, le gagnant.}. Tant que le gagnant n'a pas diffusé cette transaction, il ne peut pas récupérer les mises.

Ainsi pour vérifier l'appartenance d'une bille, les nodes doivent uniquement parcourir la liste des \textit{claim transactions} au lieu de recalculer l'état de la partie et vérifier que les billes se touchent. 
Cela permet également de constituer rapidement l'historique d'appartenance de la bille. 

Seul le gagnant peut diffuser cette transacion, et celui-ci peut décider d'envoyer ses gains à soi-même, ou à quelqu'un d'autre.

\subsection{Nodes}
Une \textit{node} est un ordinateur connecté à tout le réseau de la blockchain. Elle a un rôle majeur sur la sécurité du réseau.

\subsubsection{Arbitre}
La node possède le rôle d'arbitre.
Lorsque une partie est créée ou un joueur lance sa bille, la node va vérifier si le joueur possède bien cette bille, si c'est bien au tour du joueur de jouer, s'il n'essaye pas de lancer sa bille plus fort que la limite, etc.
Si le contrat est valide, celui-ci rentre dans la \textit{memory pool} de la node, sinon il est rejeté.

\subsubsection{Régistre}
La node s'occupe de stocker l'historique entier de toutes les parties et transactions de tous les joueurs, c'est-à-dire la blockchain.

L'ordinateur réserve également un emplacement pour stocker les transactions non confirmées par les mineurs, c'est-à-dire les lancés de billes et créations de parties qui viennent d'être diffusées par les joueurs. Cet emplacement s'appelle la \textit{memory pool}.

\subsubsection{Distribution}
Le rôle principal de la node est de partager la blockchain à toutes les autres nodes du réseau. 
La node qui possède la plus grande chaîne est celle qui possède la chaîne la plus valide, elle doit donc la partager au réseau.

\subsection{Mineurs}
Les mineurs assurent l'immuabilité de la blockchain.
Un block contient des contrats et des lancés de billes. Pour ajouter un block à la blockchain, un puzzle constitué exclusivement des données du block doit être résolu. Ce puzzle n'est solvable uniquement en essayant des milliards de combinaisons par secondes.

\subsubsection{Block hash}

Un hash\footnote{Une fonction hash est une fonction qui prend une entrée telle que du texte ou un nombre, et retourne un nombre qui apparaît aléatoire mais est déterministe. Pour la même entrée, la fonction retournera toujours la même sortie. C'est une fonction à sens unique, le hash ne permet pas de retrouver l'entrée.} SHA-256 est un nombre hexadécimal aléatoire compris entre 1 et $2^{256} - 1$.

La solution du puzzle que les mineurs doivent résoudre est en réalité un hash du block qui doit être compris entre 1 et $\frac{2^{256}-1}{mining\ difficulty}$.

Les mineurs forment un block à partir des contrats de la \textit{memory pool} des nodes.

Si le hash n'est pas compris dans cet intervalle, les mineurs calculent à nouveau un hash du block en concatenant un \textit{nonce} différent. C'est un nombre arbitraire qui permet uniquement de générer un hash différent en conservant les valeurs du block.

Ce hash sécurise la blockchain car il est facile de le résoudre si tout le réseau travail sur ce problème, mais il est impossible de résoudre ce problème seul. Ce système assume que plus de 50\% du réseau est honnête.

Si un utilisateur malveillant voulait essayer de falsifier une transaction en l'ajoutant soi-même à la blockchain, c'est-à-dire en minant le block soi-même, il devrait alors résoudre ce puzzle extrêmement compliqué seul. De plus, aucune node n'accepterai cette transaction, sauf si l'utilisateur possède plus de la moitié des ordinateurs du réseau (\textit{51\% attack}) car l'utilisateur pourrait ainsi corrompre le consensus.
Cette attaque est extrêmement peu probable.

\subsubsection{Ajustement de la difficulté}

La difficulté de minage est une variable qui assure que les mineurs ajoutent des blocks à la blockchain de manière régulière même si le nombre de mineurs, donc de puissance de calcul, augmente.
 
Cette difficulté permet aux nodes de prévoir l’espace disque nécessaire pour stocker la blockchain et assure une économie stable (voir section suivante). Elle doit donc être ajustée régulièrement.

Le \textit{block time} est une constante qui définit le temps qu’un mineur met pour ajouter un block à la chaîne.

Sur la plateforme de jeu de billes, le block time doit être assez court (entre 20s et une minute) afin de permettre des exécutions de contrats (créations de parties, et lancés de billes) rapides. Par exemple, le Bitcoin utilise un block time de 10 minutes, et Ethereum un block time de 20 secondes.

Un intervalle trop court peut causer des problèmes sur le délai de distribution des nouveaux blocks entre les nodes.

La difficulté de minage est au départ 1. Celle-ci est ajustée tous les X
blocks. Chaque node va calculer le temps que X block prennent pour être minés en théorie, divisé par la moyenne de temps que les X derniers blocks ont pris pour être minés. La difficulté est ainsi multipliée par ce coefficient.

\[ nouvelle\ difficult\acute{e} = difficult\acute{e} \times \frac{temps\ attendu}{temps\ r\acute{e}el} \]

\subsubsection{Récompenses}
Dans chaque block miné, le mineur inclut une transaction spéciale : la \textit{coinbase transaction}. C'est la récompense du mineur.
Cette transaction n'a pas d'input, uniquement l'adresse publique du mineur pour recevoir sa récompense.
Celle-ci contient de nouvelles billes créées, toutes plus ou moins rares telles que 1000 billes grises, 100 billes bleues, et 50 billes rouges. Les rouges seraient par définition plus rares que les autres car il y en aurait moins en circulation.
En plus de ces nouvelles billes, le mineur récupère toutes les taxes des transactions de chaque joueur (sous la forme de billes également).

Tous les X blocks (210 000 blocks pour Bitcoin $\approx$ 4\ ans), les récompenses des mineurs sont divisées par deux jusqu'à que la totalité des billes soient distribuées.
La prochaine récompense serait de 500 billes grises, 50 billes bleues, et 25 billes rouges.

\section{Gameplay}




\section{Outils utilisés}
Nous allons utiliser de nombreux outils durant le développement de ce projet.

\subsection{Développement global}
Les principaux outils que nous utiliserons seront :
\begin{itemize}
    \item C\#
    \item Git
    \item GitHub
    \item \LaTeX
    \item Discord
    \item Google

\end{itemize}

\subsection{Développement de la blockchain}
Nous utiliserons différents éditeurs de code tels que VIM, VS Code, ou Rider.

Nous utiliserons les technologies suivantes pour le développement de la blockchain :
\begin{itemize}
\begin{samepage}
    \item LevelDB : base de données clé/valeur
    \item ECDSA : algorithme de signature digitale
    \item SHA-256 : algorithme de hash
\end{samepage}
\end{itemize}

\subsection{Développement du jeu}

Les logiciels suivants nous aideront durant le développement du jeu :
\begin{itemize}
    \item Unity
    \item Adobe Photoshop
    \item Blender
\end{itemize}

\subsection{Developpement du site web}
Les technologies suivantes nous aideront durant le développement du site web :
\begin{itemize}
    \item Netlify
    \item React.js
    \item Sass
    \item MongoDB / MySQL
    \item Google Chrome
\end{itemize}


\end{document}
